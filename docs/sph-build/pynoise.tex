%% Generated by Sphinx.
\def\sphinxdocclass{report}
\documentclass[letterpaper,10pt,english]{sphinxmanual}
\ifdefined\pdfpxdimen
   \let\sphinxpxdimen\pdfpxdimen\else\newdimen\sphinxpxdimen
\fi \sphinxpxdimen=.75bp\relax

\PassOptionsToPackage{warn}{textcomp}
\usepackage[utf8]{inputenc}
\ifdefined\DeclareUnicodeCharacter
% support both utf8 and utf8x syntaxes
\edef\sphinxdqmaybe{\ifdefined\DeclareUnicodeCharacterAsOptional\string"\fi}
  \DeclareUnicodeCharacter{\sphinxdqmaybe00A0}{\nobreakspace}
  \DeclareUnicodeCharacter{\sphinxdqmaybe2500}{\sphinxunichar{2500}}
  \DeclareUnicodeCharacter{\sphinxdqmaybe2502}{\sphinxunichar{2502}}
  \DeclareUnicodeCharacter{\sphinxdqmaybe2514}{\sphinxunichar{2514}}
  \DeclareUnicodeCharacter{\sphinxdqmaybe251C}{\sphinxunichar{251C}}
  \DeclareUnicodeCharacter{\sphinxdqmaybe2572}{\textbackslash}
\fi
\usepackage{cmap}
\usepackage[T1]{fontenc}
\usepackage{amsmath,amssymb,amstext}
\usepackage{babel}
\usepackage{times}
\usepackage[Bjarne]{fncychap}
\usepackage{sphinx}

\fvset{fontsize=\small}
\usepackage{geometry}

% Include hyperref last.
\usepackage{hyperref}
% Fix anchor placement for figures with captions.
\usepackage{hypcap}% it must be loaded after hyperref.
% Set up styles of URL: it should be placed after hyperref.
\urlstyle{same}
\addto\captionsenglish{\renewcommand{\contentsname}{Contents:}}

\addto\captionsenglish{\renewcommand{\figurename}{Fig.}}
\addto\captionsenglish{\renewcommand{\tablename}{Table}}
\addto\captionsenglish{\renewcommand{\literalblockname}{Listing}}

\addto\captionsenglish{\renewcommand{\literalblockcontinuedname}{continued from previous page}}
\addto\captionsenglish{\renewcommand{\literalblockcontinuesname}{continues on next page}}
\addto\captionsenglish{\renewcommand{\sphinxnonalphabeticalgroupname}{Non-alphabetical}}
\addto\captionsenglish{\renewcommand{\sphinxsymbolsname}{Symbols}}
\addto\captionsenglish{\renewcommand{\sphinxnumbersname}{Numbers}}

\addto\extrasenglish{\def\pageautorefname{page}}

\setcounter{tocdepth}{1}



\title{pynoise Documentation}
\date{Oct 17, 2018}
\release{v0.3.2}
\author{Marco Guerreiro}
\newcommand{\sphinxlogo}{\vbox{}}
\renewcommand{\releasename}{Release}
\makeindex
\begin{document}

\pagestyle{empty}
\maketitle
\pagestyle{plain}
\sphinxtableofcontents
\pagestyle{normal}
\phantomsection\label{\detokenize{index::doc}}



\chapter{Introduction}
\label{\detokenize{intro:introduction}}\label{\detokenize{intro::doc}}
This package was created to add noise to signals, which can be matrices or vectors. As of October 2018, only the additive white gaussian noise is implemented.


\section{Additive White Gaussian Noise}
\label{\detokenize{intro:additive-white-gaussian-noise}}
Additive White Gaussian Noise (AWGN) is a signal generated from a normal distribution. Due to this characteristic, the noise energy is spread across all spectrum.


\chapter{pynoise package}
\label{\detokenize{pynoise:pynoise-package}}\label{\detokenize{pynoise::doc}}

\section{Submodules}
\label{\detokenize{pynoise:submodules}}

\section{pynoise.noise module}
\label{\detokenize{pynoise:module-pynoise.noise}}\label{\detokenize{pynoise:pynoise-noise-module}}\index{pynoise.noise (module)}
This module contains functions to work with signals and noise.

\phantomsection\label{\detokenize{pynoise:module-noise}}\index{noise (module)}\index{awgn() (in module pynoise.noise)}

\begin{fulllineitems}
\phantomsection\label{\detokenize{pynoise:pynoise.noise.awgn}}\pysiglinewithargsret{\sphinxcode{\sphinxupquote{pynoise.noise.}}\sphinxbfcode{\sphinxupquote{awgn}}}{\emph{x}, \emph{snr}, \emph{out='signal'}}{}
Adds White Gaussian Noise to a signal.

The noise level is specified as a Signal-to-Noise Ratio (SNR) value.
The SNR is defined as:
\begin{equation*}
\begin{split}SNR = 10\log_{10}\left(\frac{E_x}{E_n}\right)\end{split}
\end{equation*}
where \(E_x\) is the signal power and \(E_n\) is the noise
power. The noise of a discrete signal can be computed as:
\begin{equation*}
\begin{split}E = \frac{1}{N}\sum_{k=0}^{N - 1}|x_k|^2\end{split}
\end{equation*}\begin{quote}\begin{description}
\item[{Parameters}] \leavevmode\begin{itemize}
\item {} 
\sphinxstyleliteralstrong{\sphinxupquote{x}} (\sphinxcode{\sphinxupquote{np.ndarray}}) \textendash{} Signal.

\item {} 
\sphinxstyleliteralstrong{\sphinxupquote{snr}} (\sphinxstyleliteralemphasis{\sphinxupquote{int}}\sphinxstyleliteralemphasis{\sphinxupquote{, }}\sphinxstyleliteralemphasis{\sphinxupquote{float}}) \textendash{} Signal-to-Noise ration.

\item {} 
\sphinxstyleliteralstrong{\sphinxupquote{out}} (\sphinxstyleliteralemphasis{\sphinxupquote{str}}\sphinxstyleliteralemphasis{\sphinxupquote{, }}\sphinxstyleliteralemphasis{\sphinxupquote{optional}}) \textendash{} Output data. If ‘signal’, the signal \sphinxtitleref{x} plus noise is
returned. If ‘noise’, only the noise vector is returned. If
‘both’, signal with noise and noise only are returned. Any
other value defaults to ‘signal’.

\end{itemize}

\item[{Returns}] \leavevmode
Corrupted signal.

\item[{Return type}] \leavevmode
\sphinxcode{\sphinxupquote{np.ndarray}}

\end{description}\end{quote}

\end{fulllineitems}



\section{Module contents}
\label{\detokenize{pynoise:module-pynoise}}\label{\detokenize{pynoise:module-contents}}\index{pynoise (module)}

\chapter{Indices and tables}
\label{\detokenize{index:indices-and-tables}}\begin{itemize}
\item {} 
\DUrole{xref,std,std-ref}{genindex}

\item {} 
\DUrole{xref,std,std-ref}{modindex}

\item {} 
\DUrole{xref,std,std-ref}{search}

\end{itemize}


\renewcommand{\indexname}{Python Module Index}
\begin{sphinxtheindex}
\let\bigletter\sphinxstyleindexlettergroup
\bigletter{n}
\item\relax\sphinxstyleindexentry{noise}\sphinxstyleindexpageref{pynoise:\detokenize{module-noise}}
\indexspace
\bigletter{p}
\item\relax\sphinxstyleindexentry{pynoise}\sphinxstyleindexpageref{pynoise:\detokenize{module-pynoise}}
\item\relax\sphinxstyleindexentry{pynoise.noise}\sphinxstyleindexpageref{pynoise:\detokenize{module-pynoise.noise}}
\end{sphinxtheindex}

\renewcommand{\indexname}{Index}
\printindex
\end{document}