%% Generated by Sphinx.
\def\sphinxdocclass{report}
\documentclass[letterpaper,10pt,english]{sphinxmanual}
\ifdefined\pdfpxdimen
   \let\sphinxpxdimen\pdfpxdimen\else\newdimen\sphinxpxdimen
\fi \sphinxpxdimen=.75bp\relax

\PassOptionsToPackage{warn}{textcomp}
\usepackage[utf8]{inputenc}
\ifdefined\DeclareUnicodeCharacter
% support both utf8 and utf8x syntaxes
\edef\sphinxdqmaybe{\ifdefined\DeclareUnicodeCharacterAsOptional\string"\fi}
  \DeclareUnicodeCharacter{\sphinxdqmaybe00A0}{\nobreakspace}
  \DeclareUnicodeCharacter{\sphinxdqmaybe2500}{\sphinxunichar{2500}}
  \DeclareUnicodeCharacter{\sphinxdqmaybe2502}{\sphinxunichar{2502}}
  \DeclareUnicodeCharacter{\sphinxdqmaybe2514}{\sphinxunichar{2514}}
  \DeclareUnicodeCharacter{\sphinxdqmaybe251C}{\sphinxunichar{251C}}
  \DeclareUnicodeCharacter{\sphinxdqmaybe2572}{\textbackslash}
\fi
\usepackage{cmap}
\usepackage[T1]{fontenc}
\usepackage{amsmath,amssymb,amstext}
\usepackage{babel}
\usepackage{times}
\usepackage[Bjarne]{fncychap}
\usepackage[,numfigreset=1,mathnumfig]{sphinx}

\fvset{fontsize=\small}
\usepackage{geometry}

% Include hyperref last.
\usepackage{hyperref}
% Fix anchor placement for figures with captions.
\usepackage{hypcap}% it must be loaded after hyperref.
% Set up styles of URL: it should be placed after hyperref.
\urlstyle{same}
\addto\captionsenglish{\renewcommand{\contentsname}{Contents:}}

\addto\captionsenglish{\renewcommand{\figurename}{Fig.}}
\addto\captionsenglish{\renewcommand{\tablename}{Table}}
\addto\captionsenglish{\renewcommand{\literalblockname}{Listing}}

\addto\captionsenglish{\renewcommand{\literalblockcontinuedname}{continued from previous page}}
\addto\captionsenglish{\renewcommand{\literalblockcontinuesname}{continues on next page}}
\addto\captionsenglish{\renewcommand{\sphinxnonalphabeticalgroupname}{Non-alphabetical}}
\addto\captionsenglish{\renewcommand{\sphinxsymbolsname}{Symbols}}
\addto\captionsenglish{\renewcommand{\sphinxnumbersname}{Numbers}}

\addto\extrasenglish{\def\pageautorefname{page}}

\setcounter{tocdepth}{1}



\title{pynoise Documentation}
\date{Oct 25, 2018}
\release{v0.3.4}
\author{Marco Guerreiro}
\newcommand{\sphinxlogo}{\vbox{}}
\renewcommand{\releasename}{Release}
\makeindex
\begin{document}

\pagestyle{empty}
\maketitle
\pagestyle{plain}
\sphinxtableofcontents
\pagestyle{normal}
\phantomsection\label{\detokenize{index::doc}}



\chapter{Introduction}
\label{\detokenize{intro:introduction}}\label{\detokenize{intro::doc}}
This package was created to add noise to signals, which can be matrices or vectors. As of October 2018, only the additive white gaussian noise is implemented.


\section{Additive White Gaussian Noise}
\label{\detokenize{intro:additive-white-gaussian-noise}}
Additive White Gaussian Noise (AWGN) is a signal generated from a normal distribution. Hence, the noise energy is spread across all spectrum.

\hyperref[\detokenize{intro:fig-noise-spectra}]{Fig.\@ \ref{\detokenize{intro:fig-noise-spectra}}} shows the histogram (top) and spectra (bottom) of a typical noise signal.

\begin{figure}[htbp]
\centering
\capstart

\noindent\sphinxincludegraphics[scale=0.5]{{noise_spectra}.pdf}
\caption{Noise histogram (top) and spectra (bottom).}\label{\detokenize{intro:id1}}\label{\detokenize{intro:fig-noise-spectra}}\end{figure}


\chapter{Examples}
\label{\detokenize{examples:examples}}\label{\detokenize{examples::doc}}

\section{Corrupt ramp signal}
\label{\detokenize{examples:corrupt-ramp-signal}}
Add 30 dB of white gaussian noise to the ramp signal \(x(t) = t\):

\fvset{hllines={, ,}}%
\begin{sphinxVerbatim}[commandchars=\\\{\}]
\PYG{k+kn}{import} \PYG{n+nn}{numpy} \PYG{k+kn}{as} \PYG{n+nn}{np}
\PYG{k+kn}{import} \PYG{n+nn}{pynoise}
\PYG{k+kn}{import} \PYG{n+nn}{matplotlib.pyplot} \PYG{k+kn}{as} \PYG{n+nn}{plt}

\PYG{c+c1}{\PYGZsh{} \PYGZhy{}\PYGZhy{}\PYGZhy{} Signal \PYGZhy{}\PYGZhy{}\PYGZhy{}}
\PYG{n}{t} \PYG{o}{=} \PYG{n}{np}\PYG{o}{.}\PYG{n}{arange}\PYG{p}{(}\PYG{l+m+mi}{0}\PYG{p}{,} \PYG{l+m+mi}{1}\PYG{p}{,} \PYG{l+m+mf}{0.01}\PYG{p}{)}

\PYG{n}{x} \PYG{o}{=} \PYG{n}{t}
\PYG{n}{xn} \PYG{o}{=} \PYG{n}{pynoise}\PYG{o}{.}\PYG{n}{awgn}\PYG{p}{(}\PYG{n}{x}\PYG{p}{,} \PYG{l+m+mi}{30}\PYG{p}{)}

\PYG{c+c1}{\PYGZsh{} \PYGZhy{}\PYGZhy{}\PYGZhy{} Plots \PYGZhy{}\PYGZhy{}\PYGZhy{}}
\PYG{n}{plt}\PYG{o}{.}\PYG{n}{figure}\PYG{p}{(}\PYG{n}{figsize}\PYG{o}{=}\PYG{p}{(}\PYG{l+m+mi}{10}\PYG{p}{,}\PYG{l+m+mi}{6}\PYG{p}{)}\PYG{p}{)}
\PYG{n}{plt}\PYG{o}{.}\PYG{n}{plot}\PYG{p}{(}\PYG{n}{t}\PYG{p}{,} \PYG{n}{x}\PYG{p}{,} \PYG{n}{label}\PYG{o}{=}\PYG{l+s+s1}{\PYGZsq{}}\PYG{l+s+s1}{Original signal}\PYG{l+s+s1}{\PYGZsq{}}\PYG{p}{)}
\PYG{n}{plt}\PYG{o}{.}\PYG{n}{plot}\PYG{p}{(}\PYG{n}{t}\PYG{p}{,} \PYG{n}{xn}\PYG{p}{,}  \PYG{n}{label}\PYG{o}{=}\PYG{l+s+s1}{\PYGZsq{}}\PYG{l+s+s1}{Corrupted signal}\PYG{l+s+s1}{\PYGZsq{}}\PYG{p}{)}
\PYG{n}{plt}\PYG{o}{.}\PYG{n}{grid}\PYG{p}{(}\PYG{p}{)}
\PYG{n}{plt}\PYG{o}{.}\PYG{n}{xlabel}\PYG{p}{(}\PYG{l+s+s1}{\PYGZsq{}}\PYG{l+s+s1}{t}\PYG{l+s+s1}{\PYGZsq{}}\PYG{p}{)}
\PYG{n}{plt}\PYG{o}{.}\PYG{n}{ylabel}\PYG{p}{(}\PYG{l+s+s1}{\PYGZsq{}}\PYG{l+s+s1}{x}\PYG{l+s+s1}{\PYGZsq{}}\PYG{p}{)}
\PYG{n}{plt}\PYG{o}{.}\PYG{n}{legend}\PYG{p}{(}\PYG{p}{)}
\PYG{n}{plt}\PYG{o}{.}\PYG{n}{show}\PYG{p}{(}\PYG{p}{)}
\end{sphinxVerbatim}

\begin{figure}[htbp]
\centering

\noindent\sphinxincludegraphics[scale=0.8]{{noise_ex_app}.pdf}
\end{figure}


\chapter{pynoise package}
\label{\detokenize{pynoise:pynoise-package}}\label{\detokenize{pynoise::doc}}

\section{Submodules}
\label{\detokenize{pynoise:submodules}}

\section{pynoise.noise module}
\label{\detokenize{pynoise:module-pynoise.noise}}\label{\detokenize{pynoise:pynoise-noise-module}}\index{pynoise.noise (module)}
This module contains functions to work with signals and noise.

\phantomsection\label{\detokenize{pynoise:module-noise}}\index{noise (module)}\index{awgn() (in module pynoise.noise)}

\begin{fulllineitems}
\phantomsection\label{\detokenize{pynoise:pynoise.noise.awgn}}\pysiglinewithargsret{\sphinxcode{\sphinxupquote{pynoise.noise.}}\sphinxbfcode{\sphinxupquote{awgn}}}{\emph{x}, \emph{snr}, \emph{out='signal'}, \emph{method='vectorized'}}{}
Adds White Gaussian Noise to a signal.

The noise level is specified as a Signal-to-Noise Ratio (SNR) value,
which relates to signal-to-noise energy or power.
\begin{quote}\begin{description}
\item[{Parameters}] \leavevmode\begin{itemize}
\item {} 
\sphinxstyleliteralstrong{\sphinxupquote{x}} (\sphinxtitleref{np.ndarray}) \textendash{} Signal, as a vector or column-matrix.

\item {} 
\sphinxstyleliteralstrong{\sphinxupquote{snr}} (\sphinxstyleliteralemphasis{\sphinxupquote{int}}\sphinxstyleliteralemphasis{\sphinxupquote{, }}\sphinxstyleliteralemphasis{\sphinxupquote{float}}) \textendash{} Signal-to-Noise ration.

\item {} 
\sphinxstyleliteralstrong{\sphinxupquote{out}} (\sphinxstyleliteralemphasis{\sphinxupquote{str}}\sphinxstyleliteralemphasis{\sphinxupquote{, }}\sphinxstyleliteralemphasis{\sphinxupquote{optional}}) \textendash{} Output data. If ‘signal’, the signal \sphinxtitleref{x} plus noise is
returned. If ‘noise’, only the noise vector is returned. If
‘both’, signal with noise and noise only are returned. Any
other value defaults to ‘signal’.

\item {} 
\sphinxstyleliteralstrong{\sphinxupquote{method}} (\sphinxstyleliteralemphasis{\sphinxupquote{str}}\sphinxstyleliteralemphasis{\sphinxupquote{, }}\sphinxstyleliteralemphasis{\sphinxupquote{optional}}) \textendash{} Method to compute noise vector (matrix) to be introduced in the
signal. In the ‘vectorized’ method, the matrix energy is computed
and used to compute the noise energy. In the ‘max\_en’ method, the
energy of each column of \sphinxtitleref{x} is computed and only the highest
value is used to compute the noise energy. The ‘vectorized’ method
is used by default.

\end{itemize}

\item[{Returns}] \leavevmode
Corrupted signal.

\item[{Return type}] \leavevmode
\sphinxtitleref{np.ndarray}

\item[{Raises}] \leavevmode
\sphinxcode{\sphinxupquote{ValueErrorExpection}} \textendash{} If \sphinxtitleref{method} is not recognized.

\end{description}\end{quote}

\end{fulllineitems}



\section{Module contents}
\label{\detokenize{pynoise:module-pynoise}}\label{\detokenize{pynoise:module-contents}}\index{pynoise (module)}

\chapter{pynoise}
\label{\detokenize{modules:pynoise}}\label{\detokenize{modules::doc}}

\chapter{Indices and tables}
\label{\detokenize{index:indices-and-tables}}\begin{itemize}
\item {} 
\DUrole{xref,std,std-ref}{genindex}

\item {} 
\DUrole{xref,std,std-ref}{modindex}

\item {} 
\DUrole{xref,std,std-ref}{search}

\end{itemize}


\renewcommand{\indexname}{Python Module Index}
\begin{sphinxtheindex}
\let\bigletter\sphinxstyleindexlettergroup
\bigletter{n}
\item\relax\sphinxstyleindexentry{noise}\sphinxstyleindexpageref{pynoise:\detokenize{module-noise}}
\indexspace
\bigletter{p}
\item\relax\sphinxstyleindexentry{pynoise}\sphinxstyleindexpageref{pynoise:\detokenize{module-pynoise}}
\item\relax\sphinxstyleindexentry{pynoise.noise}\sphinxstyleindexpageref{pynoise:\detokenize{module-pynoise.noise}}
\end{sphinxtheindex}

\renewcommand{\indexname}{Index}
\printindex
\end{document}